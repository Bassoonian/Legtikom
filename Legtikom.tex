\documentclass[paper=6in:9in]{scrbook}
%behind the scenes
\usepackage{fontspec}
\usepackage{color}
\usepackage{calc}

%looks
\renewcommand*\sectfont{\normalcolor\bfseries}
\usepackage[hidelinks]{hyperref}

%tables
\usepackage{tabu}
\usepackage{multirow}
\usepackage{multicol}
\usepackage{float}
\restylefloat{table}

%font stuff
\setmainfont[Ligatures=TeX]{Charis SIL}
\newfontfamily\lib{Linux Libertine}

%macros
\usepackage{xebolpie}
\newcommand{\pie}[1]{\PIE{#1}{circumflex, hx}}
\newcommand{\dicttitle}[1]{\PIE{#1}{circumflex, hx, returnvar}\section*{*\apple}}
\newcommand{\formation}[5]{\noindent\-\textsc{#1} #2 \textit{#3} #4 #5}
\newcommand{\subformation}[5]{\\\noindent\-\hspace{2em}\begin{minipage}{\linewidth}\textsc{#1} #2 \textit{#3} #4 #5\end{minipage}}

\title{\PIE{Leg'tik''om}{noasterisk, circumflex}}
\author{A Proto-Indo-European Dictionary}
\date{Author}

\begin{document}
\frontmatter
\maketitle

\newpage

\tableofcontents
\newpage

\chapter{Preface}

\chapter{Abbreviations and Symbols}

\mainmatter
\chapter{Introduction}

\part{Grammar}

\part{Dictionary}

\chapter{Roots}

\dicttitle{(s)kand-}
\begin{enumerate}
\item to shine
\end{enumerate}
\noindent\textbf{Unknown formation:}\\
\formation{alb}{hënë}{moon}{!}{}
\formation{san}{cándra}{moon}{!}{}

\dicttitle{(s)m(e)wg(h)-}
\begin{enumerate}
\item smoke
\end{enumerate}
\noindent\textbf{Unknown formation:}\\
\formation{eng}{smoke}{}{!}{}
\formation{ell}{smū́khō}{burn in a smouldering fire}{!}{}
\formation{arm}{mux}{smoke}{!}{}

\dicttitle{(s)mel-}
\begin{enumerate}
\item to give off light smoke
\item to smoulder
\end{enumerate}
\noindent\textbf{Unknown formation:}\\
\formation{mga}{smāl\textasciitilde smōl\textasciitilde smūal}{fire; glow; ashes}{!}{}
\formation{eng}{smoulder}{}{!}{}
\formation{eng}{smell}{}{!}{}
\formation{lit}{smilëkti}{to give off light dust or smoke}{!}{}
\formation{wen}{smališ}{singe}{!}{}
\formation{xtb}{meli}{nose}{!}{}

\dicttitle{(s)meld-}
\begin{enumerate}
\item to melt
\end{enumerate}
\noindent\textbf{Unknown formation:}\\
\formation{eng}{melt}{}{!}{}
\formation{ell}{méldomai}{melt}{!}{}

\dicttitle{(s)mex1l-}
\begin{enumerate}
\item small animal
\end{enumerate}
\noindent\textbf{Unknown formation:}\\
\formation{sga}{mīl}{(small) animal}{!}{}
\formation{dut}{maal}{young cow}{!}{}
\formation{eng}{small}{}{!}{}
\formation{ell}{mēlon}{sheep goat}{!}{}

\dicttitle{(s)tenxx-}
\begin{enumerate}
\item to groan
\item to thunder
\end{enumerate}
\noindent\textbf{Unknown formation:}\\
\formation{lat}{tonāre}{to thunder}{!}{}
\formation{ang}{þunor}{thunder}{!}
{\subformation{eng}{thunder}{}{!}{}
}\formation{chu}{stenǫ}{groan}{!}{}
\formation{ell}{stēnō}{thunder}{!}{}
\formation{san}{stanáyati}{thunders}{!}{}

\dicttitle{(s)ter-}
\begin{enumerate}
\item stork
\end{enumerate}
\noindent\textbf{Unknown formation:}\\
\formation{eng}{stork}{}{!}{}
\formation{hit}{tarlā}{stork}{!}{}

\dicttitle{(y)ebh-}
\begin{enumerate}
\item elephant
\item ivory
\end{enumerate}
\noindent\textbf{Unknown formation:}\\
\formation{lat}{ebur}{}{!}{}
\formation{san}{íbha-}{}{!}{}

\dicttitle{b(e)w-}
\begin{enumerate}
\item owl
\end{enumerate}
\noindent\textbf{Unknown formation:}\\
\formation{lat}{būbō}{owl}{!}{}
\formation{bul}{buk}{owl}{!}{}
\formation{ell}{búas}{owl}{!}{}
\formation{arm}{bu\textasciitilde bueč}{owl}{!}{}
\formation{per}{būm}{owl}{!}{}

\dicttitle{bhel-}
\begin{enumerate}
\item coot
\end{enumerate}
\noindent\textbf{Unknown formation:}\\
\formation{lat}{fulica}{coot}{!}{}
\formation{goh}{belihha}{}{!}{}
\formation{ell}{phalarís}{}{!}{}

\dicttitle{bherg'h-}
\begin{enumerate}
\item hill
\item mountain
\end{enumerate}
\noindent\textbf{Unknown formation:}\\
\formation{mga}{brī}{hill}{!}{}
\formation{eng}{barrow}{}{!}{}
\formation{eng}{borrough}{}{!}{}
\formation{ger}{Berg}{mountain}{!}{}
\formation{rus}{béreg}{river-bank}{!}{}
\formation{ave}{bərəz-}{hill}{!}{}

\dicttitle{demxA-}
\begin{enumerate}
\item to tame
\item to subdue
\end{enumerate}
\noindent\textbf{Unknown formation:}\\
\formation{sga}{damnaid}{binds; breaks (a horse)}{!}{}
\formation{sga}{dam}{ox}{!}{}
\formation{lat}{domō}{break tame}{!}{}
\formation{eng}{tame}{}{!}{}
\formation{ell}{dámnēmi}{break}{!}{}
\formation{hit}{damaszi}{presses; pushes}{!}{}
\formation{per}{dām}{tamed animal}{!}{}
\formation{san}{dāmāyati}{subdues}{!}{}
\formation{san}{damitár-}{horse breaker}{!}{}

\dicttitle{dexAw-}
\begin{enumerate}
\item to kindle
\item to burn
\item to get fire started
\end{enumerate}
\noindent\textbf{Unknown formation:}\\
\formation{sga}{doud}{burning}{!}{}
\formation{ell}{daiō}{to kindle; to burn}{!}{}
\formation{san}{dunóti}{to kindle; to burn}{!}{}
\formation{xto}{twās-}{to kindle; to ignite; to light}{!}{}

\dicttitle{dey-}
\begin{enumerate}
\item to shine
\item sky
\end{enumerate}
\noindent\textbf{Unknown formation:}\\
\formation{lat}{dīum}{sky}{!}{}
\formation{san}{dyáuṣ}{sky}{!}{}

\dicttitle{dhegvh-}
\begin{enumerate}
\item to burn
\end{enumerate}
\noindent\textbf{Unknown formation:}\\
\formation{sga}{daig}{flame}{!}{}
\formation{lat}{foveō}{to heat; to cherish}{!}{}
\formation{lit}{degù}{to burn}{!}{}
\formation{chu}{žegǫ}{to burn}{!}{}
\formation{alb}{djeg}{to burn}{!}{}
\formation{alb}{ndez}{to kindle}{!}{}
\formation{ell}{téphrā}{ash}{!}{}
\formation{ave}{dazaiti}{burns}{!}{}
\formation{san}{dáhaiti}{burns}{!}{}
\formation{xto}{tsäk-}{burn}{!}{}
\formation{txb}{tsäk-}{burn}{!}{}
\formation{pgmc}{?dagaz}{day}{!}{}
{\subformation{eng}{day}{}{!}{}
}
\dicttitle{dhreg-}
\begin{enumerate}
\item to snow lightly
\item to rain lightly
\end{enumerate}
\noindent\textbf{Unknown formation:}\\
\formation{eng}{dark}{}{!}{}
\formation{lit}{dérgti}{to be blushy; to be sleety}{!}{}
\formation{orv}{padorog}{stormy weather}{!}{}
\formation{txb}{tarkër}{cloud}{!}{}

\dicttitle{g'hor-}
\begin{enumerate}
\item young pig
\end{enumerate}
\noindent\textbf{Unknown formation:}\\
\formation{alb}{derr}{pig; hog; swine}{!}{}
\formation{ell}{khoīros}{young pig; swine}{!}{}

\dicttitle{ger-}
\begin{enumerate}
\item crane
\end{enumerate}
\noindent\textbf{Unknown formation:}\\
\formation{wel}{garan}{crane}{!}{}
\formation{lat}{grūs}{crane}{!}{}
\formation{eng}{crane}{}{!}{}
\formation{lit}{gérvé}{crane}{!}{}
\formation{rus}{žeravlĭ}{crane; goose}{!}{}
\formation{arm}{kṙunk}{crane}{!}{}
\formation{oss}{zyrnæg}{crane}{!}{}

\dicttitle{ghan-}
\begin{enumerate}
\item to gape
\item to yawn
\end{enumerate}
\dicttitle{ghel(x2.)d-}
\begin{enumerate}
\item hail
\end{enumerate}
\noindent\textbf{Unknown formation:}\\
\formation{chu}{žědica}{freezing}{!}{}
\formation{ell}{khalaza}{hail}{!}{}
\formation{per}{žāla}{hail}{!}{}

\dicttitle{gves-}
\begin{enumerate}
\item to extinguish
\end{enumerate}
\noindent\textbf{Unknown formation:}\\
\formation{lit}{gèsti}{to go out; to extinguish}{!}{}
\formation{chu}{ugasiti}{to go out; to extinguish}{!}{}
\formation{ell}{sbénnūmi}{to go out; to extinguish}{!}{}
\formation{hit}{kist-}{to go out; to extinguish}{!}{}
\formation{san}{jásate}{to go out; to extinguish}{!}{}
\formation{xtb}{kes-}{to go out; to extinguish}{!}{}

\dicttitle{gvorxx-}
\begin{enumerate}
\item hill
\item mountain
\end{enumerate}
\noindent\textbf{Unknown formation:}\\
\formation{chu}{gora}{mountain}{!}{}
\formation{alb}{gur}{rock}{!}{}
\formation{ave}{gairi-}{mountain}{!}{}
\formation{san}{girí-}{mountain}{!}{}
\formation{lit}{girià}{forest}{!}{}
\formation{ell}{?Βορέας}{northwind}{!}{}

\dicttitle{k'em-}
\begin{enumerate}
\item hornless
\end{enumerate}
\noindent\textbf{Unknown formation:}\\
\formation{san}{śáma-}{hornless}{!}{}
\formation{eng}{hind}{}{!}{}
\formation{ell}{kemás}{young deer}{!}{}
\formation{prg}{camstian}{sheep}{!}{}
\formation{prg}{camnet}{horse}{!}{}
\formation{rus}{konĭ}{horse}{!}{}

\dicttitle{k'er-}
\begin{enumerate}
\item horn
\end{enumerate}
\noindent\textbf{\pie{Ø-nom}:}\\
\formation{lat}{cornum}{horn}{!}{}
\formation{eng}{horn}{}{!}{}
\noindent\textbf{\pie{e-h2.(s)}:}\\
\formation{ell}{kéras}{horn}{!}{}
\formation{xtb}{karse}{stag}{!}{}
\noindent\textbf{\pie{e:-h2.sr.}:}\\
\formation{lat}{crābrō}{hornet}{!}{}
\formation{lit}{širšuõ}{hornet}{!}{}
\formation{xtb}{krorīya}{horn}{!}{}
\noindent\textbf{\pie{o-u}:}\\
\formation{lat}{cervus}{stag}{!}{}
\formation{lit}{kárvė}{cow}{!}{}
\formation{rus}{koróva}{cow}{!}{}
\formation{ell}{kórudos}{crested lark}{!}{}
\formation{ell}{koruphḗ}{crest}{!}{}
\formation{ave}{svra-}{horn; claw; talon}{!}{}

\dicttitle{kVr-}
\begin{enumerate}
\item crow
\end{enumerate}
\noindent\textbf{Unknown formation:}\\
\formation{lat}{corvus}{crow}{!}{}
\formation{eng}{rook}{}{!}{}
\formation{bul}{krókon}{crow}{!}{}
\formation{ell}{kóraks}{crow}{!}{}
\formation{san}{karaṭa-\textasciitilde karāva}{crow}{!}{}

\dicttitle{kap-}
\begin{enumerate}
\item hawk
\item falcon
\end{enumerate}
\noindent\textbf{Unknown formation:}\\
\formation{eng}{hawk}{}{!}{}
\formation{rus}{kóbec}{type of falcon}{!}{}

\dicttitle{kap-}
\begin{enumerate}
\item to seize
\end{enumerate}
\dicttitle{kaw-}
\begin{enumerate}
\item howl
\item owl
\end{enumerate}
\noindent\textbf{Unknown formation:}\\
\formation{wel}{cuan}{}{!}{}
\formation{goh}{hūwo}{}{!}{}

\dicttitle{kelx1-}
\begin{enumerate}
\item to incline
\end{enumerate}
\dicttitle{kenk-}
\begin{enumerate}
\item back of the knee
\item hock
\end{enumerate}
\noindent\textbf{Unknown formation:}\\
\formation{eng}{hock}{}{!}{}
\formation{lit}{kenklė̃}{hock; back of the knee}{!}{}
\formation{san}{kankāla-}{bone; skeleton}{!}{}

\dicttitle{ker-}
\begin{enumerate}
\item to burn
\end{enumerate}
\noindent\textbf{Unknown formation:}\\
\formation{got}{haúri}{coal}{!}{}
\formation{non}{hyrr}{fire}{!}{}
\formation{ang}{heorþ}{hearth}{!}
{\subformation{eng}{hearth}{}{!}{}
}\formation{lit}{kùrti}{heat}{!}{}
\formation{chu}{kuriti sę}{smoke}{!}{}
\formation{lat}{cremō}{to burn}{!}{}
\formation{san}{?kaṣāku\textasciitilde kuṣāku}{fire; sun}{!}{}

\dicttitle{kerk-}
\begin{enumerate}
\item hen
\end{enumerate}
\noindent\textbf{Unknown formation:}\\
\formation{mga}{cerc}{brood hen}{!}{}
\formation{ell}{kérkos}{rooster}{!}{}
\formation{ave}{kahrka-}{hen}{!}{}
\formation{san}{kr̥kara-}{a kind of partridge}{!}{}
\formation{san}{kr̥kavā́ku-}{rooster}{!}{}
\formation{xtb}{kraṅko}{chicken}{!}{}

\dicttitle{kewl-}
\begin{enumerate}
\item pig
\end{enumerate}
\noindent\textbf{Unknown formation:}\\
\formation{wlm}{Culhwych}{mythological figure}{!}{}
\formation{lit}{kiaũle}{pig}{!}{}

\dicttitle{kexAw-}
\begin{enumerate}
\item to burn
\end{enumerate}
\noindent\textbf{Unknown formation:}\\
\formation{txb}{kaum}{day}{!}{}
\formation{ell}{kaiō}{burn}{!}{}

\dicttitle{kvap-}
\begin{enumerate}
\item to smoke
\item to seethe
\end{enumerate}
\noindent\textbf{Unknown formation:}\\
\formation{lit}{kvapas}{breath}{!}{}
\formation{ell}{kapnós}{smoke}{!}{}

\dicttitle{lebh-}
\begin{enumerate}
\item ivory
\end{enumerate}
\noindent\textbf{Unknown formation:}\\
\formation{gmy}{e-re-pa}{}{!}{}
\formation{ell}{eléphās}{}{!}{}
\formation{hit}{hahpa-}{}{!}{}

\dicttitle{lep-}
\begin{enumerate}
\item stone
\end{enumerate}
\noindent\textbf{Unknown formation:}\\
\formation{lat}{lapis}{stone}{!}{}
\formation{ell}{lépas}{stone}{!}{}

\dicttitle{lew-}
\begin{enumerate}
\item dirt
\end{enumerate}
\noindent\textbf{Unknown formation:}\\
\formation{lat}{polluō}{soil; defile}{!}{}
\formation{ell}{lūma}{dirt}{!}{}

\dicttitle{lex1w-}
\begin{enumerate}
\item stone
\end{enumerate}
\noindent\textbf{Unknown formation:}\\
\formation{sga}{līe}{stone}{!}{}
\formation{grc}{lāas}{stone}{!}{}
\formation{grc}{léusō}{to stone}{!}{}
\formation{alb}{lerë}{rubble}{!}{}

\dicttitle{li(w)-}
\begin{enumerate}
\item lion
\end{enumerate}
\noindent\textbf{Unknown formation:}\\
\formation{rus}{lev}{lion}{!}{}
\formation{ell}{lís}{lion}{!}{}

\dicttitle{luk'-}
\begin{enumerate}
\item lynx
\end{enumerate}
\noindent\textbf{Unknown formation:}\\
\formation{sga}{lug}{lynx}{!}{}
\formation{ang}{lox}{lynx}{!}{}
\formation{lit}{lū́šis}{lynx}{!}{}
\formation{rus}{rysĭ}{lynx}{!}{}
\formation{ell}{lūgks}{lynx}{!}
{\subformation{lat}{lynx}{}{!}
{\subformation{eng}{lynx}{}{!}{}
}}\formation{arm}{lusanunk'}{lynx}{!}{}

\dicttitle{may-}
\begin{enumerate}
\item soil
\item defile
\end{enumerate}
\noindent\textbf{Unknown formation:}\\
\formation{eng}{mole}{}{!}{}
\formation{lit}{miēles}{yeast}{!}{}

\dicttitle{meldh-}
\begin{enumerate}
\item lightning
\end{enumerate}
\noindent\textbf{Unknown formation:}\\
\formation{wel}{mellt}{lightning}{!}{}
\formation{chu}{bliskŭ}{lightning}{!}{}

\dicttitle{mews-}
\begin{enumerate}
\item to steal
\end{enumerate}
\dicttitle{mex1-}
\begin{enumerate}
\item to measure
\end{enumerate}
\dicttitle{mregh-}
\begin{enumerate}
\item to rain softly
\item to drizzle
\end{enumerate}
\noindent\textbf{Unknown formation:}\\
\formation{lav}{merguōt}{to rain softly}{!}{}
\formation{ell}{brékhei}{rains}{!}{}

\dicttitle{n(o)xxt-}
\begin{enumerate}
\item rear-end
\end{enumerate}
\noindent\textbf{Unknown formation:}\\
\formation{lat}{nātis}{human bottocks}{!}{}
\formation{ell}{nōton}{back}{!}{}

\dicttitle{pad-}
\begin{enumerate}
\item duck
\end{enumerate}
\noindent\textbf{Unknown formation:}\\
\formation{spa}{pato}{duck}{!}{}
\formation{arm}{bad}{drake}{!}{}
\formation{per}{ba}{}{!}{}

\dicttitle{pe:(n)s-}
\begin{enumerate}
\item dust
\end{enumerate}
\noindent\textbf{Unknown formation:}\\
\formation{chu}{pĕsŭkŭ}{dust}{!}{}
\formation{ave}{pąsnu}{dust}{!}{}
\formation{san}{pāṃsú}{crumbling soil; sand; dust}{!}{}

\dicttitle{pe:nt-}
\begin{enumerate}
\item heel
\end{enumerate}
\noindent\textbf{Unknown formation:}\\
\formation{prg}{pentis}{heel}{!}{}
\formation{rus}{pjatá}{heel}{!}{}
\formation{pus}{pūnda}{heel}{!}{}

\dicttitle{pen-}
\begin{enumerate}
\item water
\end{enumerate}
\noindent\textbf{Unknown formation:}\\
\formation{sga}{en}{water}{!}{}
\formation{eng}{fen}{}{!}{}
\formation{prg}{pannean}{peat-bog}{!}{}

\dicttitle{penk-}
\begin{enumerate}
\item wet?
\item mud?
\end{enumerate}
\noindent\textbf{Unknown formation:}\\
\formation{ang}{fūht}{wet}{!}{}
\formation{san}{pánku-}{mud; mire}{!}{}

\dicttitle{per-}
\begin{enumerate}
\item to appear
\item to bring forth
\end{enumerate}
\noindent\textbf{Unknown formation:}\\
\formation{ang}{fearr}{bullock; steer}{!}{}
\formation{ell}{póris\textasciitilde pórtis}{calf; heifer}{!}{}
\formation{san}{pr̥thuka-}{child; young of an animal}{!}{}

\dicttitle{perd-}
\begin{enumerate}
\item panther
\item lion
\end{enumerate}
\noindent\textbf{Unknown formation:}\\
\formation{per}{palang}{}{!}{}
\formation{ell}{?párdalis}{}{!}{}

\dicttitle{perk-}
\begin{enumerate}
\item glowing ash
\item coal
\end{enumerate}
\noindent\textbf{Unknown formation:}\\
\formation{sga}{riches}{glowing coal}{!}{}
\formation{lit}{pirkšnys}{ashes with glowing sparks}{!}{}

\dicttitle{pipp-}
\begin{enumerate}
\item young bird
\end{enumerate}
\noindent\textbf{Unknown formation:}\\
\formation{slv}{pipa}{hen}{!}{}
\formation{alb}{bibë}{young bird}{!}{}
\formation{ell}{pīpos}{young bird}{!}{}
\formation{san}{pippakā-}{young bird}{!}{}

\dicttitle{prews-}
\begin{enumerate}
\item frost
\end{enumerate}
\noindent\textbf{Unknown formation:}\\
\formation{eng}{frost}{}{!}{}
\formation{lat}{pruīna}{hoarfrost}{!}{}
\formation{sga}{?reōd}{strong cold}{!}{}
\formation{san}{?prusvā́-}{hoarfrost; dew; drop}{!}{}

\dicttitle{prews-}
\begin{enumerate}
\item to burn
\end{enumerate}
\noindent\textbf{Unknown formation:}\\
\formation{lat}{prūna}{glowing coals}{!}{}
\formation{alb}{prush}{glowing}{!}{}
\formation{san}{plosati}{burns}{!}{}

\dicttitle{snewdh-}
\begin{enumerate}
\item cloud
\end{enumerate}
\noindent\textbf{Unknown formation:}\\
\formation{wel}{nudd}{mist}{!}{}
\formation{lat}{nūbēs}{cloud; mist}{!}{}
\formation{ave}{snaoδa}{cloud}{!}{}

\dicttitle{sneygvh-}
\begin{enumerate}
\item to snow
\end{enumerate}
\noindent\textbf{Unknown formation:}\\
\formation{sga}{snigid}{snows; rains}{!}{}
\formation{lat}{nivit\textasciitilde ninguit}{snows}{!}{}
\formation{ang}{snīwan}{to snow}{!}{}
\formation{ell}{neíphei}{snows}{!}{}
\formation{ave}{snaēžaiti}{snows}{!}{}

\dicttitle{solxx-}
\begin{enumerate}
\item dirt
\item dirty
\end{enumerate}
\noindent\textbf{\pie{solxx-wos}:}\\
\formation{pgmc}{salwaz}{yellowish brown}{!}
{\subformation{ang}{salo\textasciitilde salu}{dark; dusky}{!}
{\subformation{eng}{sallow}{}{!}{}
}\subformation{dum}{salu(we)}{pale; yellow; dirty}{!}{}
\subformation{goh}{salo}{dark; black; dirty}{!}{}
\subformation{non}{sǫlr}{yellow; pale}{!}{}
}\noindent\textbf{Unknown formation:}\\
\formation{ang}{sol}{dark; dirty}{!}{}
\formation{lat}{salebra}{dirt}{!}{}
\formation{txb}{sal}{dirty}{!}{}
\formation{hit}{?salpa-}{dog-dung}{!}{}

\dicttitle{sper-}
\begin{enumerate}
\item some kind of small bird
\end{enumerate}
\noindent\textbf{Unknown formation:}\\
\formation{eng}{sparrow}{}{!}{}
\formation{cor}{frau}{crow}{!}{}
\formation{ell}{sparásion}{starling}{!}{}
\formation{xto}{ṣpār}{bird}{!}{}

\dicttitle{suxx-}
\begin{enumerate}
\item rain
\end{enumerate}
\noindent\textbf{Unknown formation:}\\
\formation{ell}{húei}{rains}{!}{}
\formation{prg}{suge}{rain}{!}{}
\formation{xto}{su-}{rain}{!}{}
\formation{txb}{su-}{rain}{!}{}
\formation{alb}{?shi}{rain}{!}{}

\dicttitle{swel-}
\begin{enumerate}
\item to burn
\end{enumerate}
\noindent\textbf{Unknown formation:}\\
\formation{ang}{swelan}{burn}{!}{}
\formation{lit}{svilù}{singe}{!}{}
\formation{ell}{hélā}{heat of the sun}{!}{}

\dicttitle{swelp-}
\begin{enumerate}
\item to burn
\item to smoulder
\end{enumerate}
\noindent\textbf{Unknown formation:}\\
\formation{xto}{sälp-}{to be set alight; to burn}{!}{}
\formation{txb}{sälp-}{to be set alight; to burn}{!}{}

\dicttitle{tenxA.g-}
\begin{enumerate}
\item shallow water?
\end{enumerate}
\noindent\textbf{Unknown formation:}\\
\formation{lav}{tīgas}{deep spot in water}{!}{}
\formation{ell}{ténagos}{shoal; shallow water}{!}{}

\dicttitle{texA-}
\begin{enumerate}
\item to melt
\end{enumerate}
\noindent\textbf{Unknown formation:}\\
\formation{wel}{toddi}{melt}{!}{}
\formation{lat}{tābeō}{melt}{!}{}
\formation{eng}{thaw}{}{!}{}
\formation{chu}{tajǫ}{melt}{!}{}
\formation{ell}{tḗkō}{melt}{!}{}
\formation{arm}{t'anam}{moisten}{!}{}
\formation{oss}{tajyn\textasciitilde tajun}{melt}{!}{}
\formation{alb}{?thaj}{dry}{!}{}

\dicttitle{tixxn-}
\begin{enumerate}
\item to be dirty
\end{enumerate}
\noindent\textbf{Unknown formation:}\\
\formation{txb}{tin-}{to be dirty}{!}{}
\formation{chu}{tina}{mire; filth}{!}{}

\dicttitle{wer-}
\begin{enumerate}
\item crow
\end{enumerate}
\noindent\textbf{Unknown formation:}\\
\formation{lit}{várna}{crow}{!}{}
\formation{rus}{voróna}{crow}{!}{}
\formation{xtb}{weauña}{crow}{!}{}

\dicttitle{wexxp-}
\begin{enumerate}
\item body of water
\end{enumerate}
\noindent\textbf{Unknown formation:}\\
\formation{lit}{ùpe}{river}{!}{}
\formation{chu}{vapa}{lake}{!}{}
\formation{hit}{wappu-}{wadi; river bank}{!}{}
\formation{san}{vāpī-}{large pond}{!}{}

\dicttitle{wis-}
\begin{enumerate}
\item bison
\end{enumerate}
\noindent\textbf{Unknown formation:}\\
\formation{goh}{wisant}{bison}{!}
{\subformation{lat}{bisōn}{}{!}{}
{\subformation{eng}{bison}{}{!}{}
}}
\dicttitle{wl(o)p-}
\begin{enumerate}
\item fox
\end{enumerate}
\noindent\textbf{Unknown formation:}\\
\formation{lat}{vulpēs}{fox}{!}{}
\formation{lit}{lãpė}{fox}{!}{}
\formation{ell}{alṓpēks\textasciitilde alōpós}{fox}{!}{}
\formation{arm}{ałuēs}{fox}{!}{}
\formation{hit}{ulip(pa)na-}{wolf}{!}{}
\formation{ave}{urupis}{dog}{!}{}
\formation{ave}{raopi-}{fox; jackal}{!}{}
\formation{san}{lopāśá-}{jackal; fox}{!}{}

\dicttitle{x1eg'h-}
\begin{enumerate}
\item cow
\end{enumerate}
\noindent\textbf{Unknown formation:}\\
\formation{sga}{ag}{cow}{!}{}
\formation{arm}{ezn}{cow}{!}{}
\formation{san}{ahī-}{cow}{!}{}

\dicttitle{x1el-}
\begin{enumerate}
\item dull red
\end{enumerate}
\dicttitle{x1el-}
\begin{enumerate}
\item waterbird
\item swan
\end{enumerate}
\noindent\textbf{Unknown formation:}\\
\formation{sga}{ela}{}{!}{}
\formation{lat}{olor}{}{!}{}

\dicttitle{x1er-}
\begin{enumerate}
\item earth
\end{enumerate}
\noindent\textbf{Unknown formation:}\\
\formation{eng}{earth}{}{!}{}
\formation{ell}{erā}{earth}{!}{}

\dicttitle{x1ews-}
\begin{enumerate}
\item to burn
\item to singe
\end{enumerate}
\noindent\textbf{Unknown formation:}\\
\formation{lat}{ūrō}{burn}{!}{}
\formation{non}{ysja}{fire}{!}{}
\formation{alb}{ethe}{fever}{!}{}
\formation{ell}{heúō}{singe}{!}{}
\formation{san}{ósati}{burns; singes}{!}{}

\dicttitle{x1eyxx-}
\begin{enumerate}
\item ice
\item hoarfrost
\end{enumerate}
\noindent\textbf{Unknown formation:}\\
\formation{pgmc}{īsaz\textasciitilde īsą}{ice}{!}
{\subformation{ang}{īs}{ice}{!}
{\subformation{eng}{ice}{}{!}{}
}\subformation{non}{íss}{ice}{!}
{\subformation{fao}{ísur}{ice}{!}{}
\subformation{ovd}{ais}{ice}{!}{}
\subformation{goh}{īs}{ice}{!}{}
}}\formation{lit}{ýnis}{glazed frost}{!}{}
\formation{rus}{ínej}{hoarfrost}{!}{}
\formation{av}{aēza-}{frost; ice}{!}{}

\dicttitle{x1wers-}
\begin{enumerate}
\item rain
\end{enumerate}
\noindent\textbf{Unknown formation:}\\
\formation{ell}{eérsē}{dew}{!}{}
\formation{ell}{ouréō}{to urinate}{!}{}
\formation{hit}{warsa-}{rainfall}{!}{}
\formation{san}{vársati}{rains}{!}{}

\dicttitle{x2.wex1-}
\begin{enumerate}
\item to blow
\end{enumerate}
\dicttitle{x2eP-}
\begin{enumerate}
\item water
\item river
\item living water
\item water on the move
\end{enumerate}
\noindent\textbf{Unknown formation:}\\
\formation{sga}{ab}{river}{!}{}
\formation{wlm}{afon}{river}{!}{}
\formation{lat}{amnis}{river}{!}{}
\formation{prg}{ape}{river}{!}{}
\formation{hit}{hāpa-}{river}{!}{}
\formation{ave}{āfš-}{water}{!}{}
\formation{san}{āp-}{water}{!}{}
\formation{xto}{āp}{water}{!}{}
\formation{txb}{āp}{water}{!}{}

\dicttitle{x2exx-}
\begin{enumerate}
\item to burn
\item to be hot
\end{enumerate}
\noindent\textbf{Unknown formation:}\\
\formation{plq}{hā-}{to be hot}{!}{}

\dicttitle{x3.meygh-}
\begin{enumerate}
\item fog
\item mist
\end{enumerate}
\noindent\textbf{Unknown formation:}\\
\formation{eng}{mist}{}{!}{}
\formation{lit}{miglà}{mist}{!}{}
\formation{rus}{mgla}{mist; darkness}{!}{}
\formation{ell}{omíkhlē}{mist; fog}{!}{}
\formation{san}{meghá-}{cloud}{!}{}
\formation{alb}{mjegull}{mist; fog}{!}{}

\dicttitle{x3or-}
\begin{enumerate}
\item eagle
\end{enumerate}
\noindent\textbf{Unknown formation:}\\
\formation{sga}{irar}{eagle}{!}{}
\formation{eng}{erne}{}{!}{}
\formation{lit}{erẽlis}{eagle}{!}{}
\formation{rus}{orël}{eagle}{!}{}
\formation{hit}{hāras}{eagle}{!}{}
\formation{ell}{órnis}{bird}{!}{}
\formation{arm}{urur}{kite}{!}{}
\formation{arm}{oror}{gull}{!}{}
\formation{arm}{ori}{raven}{!}{}

\dicttitle{xAeg'-}
\begin{enumerate}
\item to drive
\end{enumerate}
\dicttitle{xAel-}
\begin{enumerate}
\item to burn
\end{enumerate}
\noindent\textbf{Unknown formation:}\\
\formation{lat}{altar}{altar}{!}{}
\formation{lat}{adoleō}{to burn up a sacrifice}{!}{}
\formation{swe}{ala}{to blaze up; to flare up}{!}{}
\formation{san}{alātam}{firebrand; coal}{!}{}

\dicttitle{xAeydh-}
\begin{enumerate}
\item to burn
\item fire
\end{enumerate}
\noindent\textbf{Unknown formation:}\\
\formation{sga}{āed}{fire}{!}{}
\formation{lat}{aedēs}{temple}{!}{}
\formation{ang}{ād}{heat; fire}{!}{}
\formation{ell}{aithō}{to burn}{!}{}
\formation{san}{indhé}{kindle}{!}{}

\dicttitle{yeg-}
\begin{enumerate}
\item ice
\item icicle
\item solid expanse of ice
\end{enumerate}
\noindent\textbf{Unknown formation:}\\
\formation{sga}{aig}{ice}{!}{}
\formation{ang}{ġiċel}{ice; icicle}{!}{}
\formation{hit}{eka-}{ice}{!}{}
\formation{srh}{yoz}{glacier}{!}{}

\dicttitle{yuxx-}
\begin{enumerate}
\item water
\end{enumerate}
\noindent\textbf{Unknown formation:}\\
\formation{lit}{jū́rės}{sea}{!}{}
\formation{txh}{iuras}{a river name}{!}{}



\chapter{Words}

\backmatter
\chapter{Index of Roots}

\chapter{Bibliography}

\begin{enumerate}
\item Kroonen, Guus. \textit{Etymological Dictionary of Proto-Germanic}. Leiden: Koninklijke Brill NV, 2013.
\item Mallory, J. P., Adams, D. Q.. \textit{The Oxford Introduction to Proto-Indo-European and the Proto-Indo-European World}. New York: Oxford University Press Inc., 2006.
\end{enumerate}

\end{document}