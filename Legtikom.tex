\documentclass[paper=6in:9in]{scrbook}
%behind the scenes
\usepackage{fontspec}
\usepackage{color}
\usepackage{calc}

%looks
\renewcommand*\sectfont{\normalcolor\bfseries}
\usepackage[hidelinks]{hyperref}

%tables
\usepackage{tabu}
\usepackage{multirow}
\usepackage{multicol}
\usepackage{float}
\restylefloat{table}

%font stuff
\setmainfont[Ligatures=TeX]{Charis SIL}
\newfontfamily\lib{Linux Libertine}

%macros
\usepackage{xebolpie}
\newcommand{\pie}[1]{\PIE{#1}{circumflex, hx}}
\newcommand{\dicttitle}[1]{\PIE{#1}{circumflex, hx, returnvar}\section*{*\apple}}
\newcommand{\formation}[5]{\noindent\-\textsc{#1} #2 \textit{#3} #4 #5\\}
\newcommand{\subformation}[5]{\\\noindent\-\hspace{2em}\begin{minipage}{\linewidth}\textsc{#1} #2 \textit{#3} #4 #5\end{minipage}}

\title{\PIE{Leg'tik''om}{noasterisk, circumflex}}
\author{A Proto-Indo-European Dictionary}
\date{Author}

\begin{document}
\frontmatter
\maketitle

\newpage

\tableofcontents
\newpage

\chapter{Preface}

\chapter{Abbreviations and Symbols}

\mainmatter
\chapter{Introduction}

\part{Grammar}

\part{Dictionary}

\chapter{Roots}

\subsubsection{\**(s)kand-}
\paragraph{Definition}
\begin{enumerate}
\item to shine
\end{enumerate}
\subsubsection{\**(s)m(e)wg(h)-}
\paragraph{Definition}
\begin{enumerate}
\item smoke
\end{enumerate}
\subsubsection{\**(s)mel-}
\paragraph{Definition}
\begin{enumerate}
\item to give off light smoke
\item to smoulder
\end{enumerate}
\subsubsection{\**(s)meld-}
\paragraph{Definition}
\begin{enumerate}
\item to melt
\end{enumerate}
\subsubsection{\**(s)tenhₓ-}
\paragraph{Definition}
\begin{enumerate}
\item to groan
\item to thunder
\end{enumerate}
\subsubsection{\**bʰerǵʰ-}
\paragraph{Definition}
\begin{enumerate}
\item hill
\item mountain
\end{enumerate}
\subsubsection{\**dehₐw-}
\paragraph{Definition}
\begin{enumerate}
\item to kindle
\item to burn
\item to get fire started
\end{enumerate}
\subsubsection{\**dey-}
\paragraph{Definition}
\begin{enumerate}
\item to shine
\end{enumerate}
\subsubsection{\**dʰegʷʰ-}
\paragraph{Definition}
\begin{enumerate}
\item to burn
\end{enumerate}
\subsubsection{\**dʰreg-}
\paragraph{Definition}
\begin{enumerate}
\item to snow lightly
\item to rain lightly
\end{enumerate}
\subsubsection{\**gʰel(h̥₂)d-}
\paragraph{Definition}
\begin{enumerate}
\item hail
\end{enumerate}
\subsubsection{\**gʷes-}
\paragraph{Definition}
\begin{enumerate}
\item to extinguish
\end{enumerate}
\subsubsection{\**gʷorhₓ-}
\paragraph{Definition}
\begin{enumerate}
\item hill
\item mountain
\end{enumerate}
\subsubsection{\**h̥₂weh₁-}
\paragraph{Definition}
\begin{enumerate}
\item to blow
\end{enumerate}
\subsubsection{\**h̥₃meygʰ-}
\paragraph{Definition}
\begin{enumerate}
\item fog
\item mist
\end{enumerate}
\subsubsection{\**h₁er-}
\paragraph{Definition}
\begin{enumerate}
\item earth
\end{enumerate}
\subsubsection{\**h₁ews-}
\paragraph{Definition}
\begin{enumerate}
\item to burn
\item to singe
\end{enumerate}
\subsubsection{\**h₁eyhₓ-}
\paragraph{Definition}
\begin{enumerate}
\item ice
\item hoarfrost
\end{enumerate}
\subsubsection{\**h₁wers-}
\paragraph{Definition}
\begin{enumerate}
\item rain
\end{enumerate}
\subsubsection{\**h₂eP-}
\paragraph{Definition}
\begin{enumerate}
\item water
\item river
\item living water
\item water on the move
\end{enumerate}
\subsubsection{\**h₂ehₓ-}
\paragraph{Definition}
\begin{enumerate}
\item to burn
\item to be hot
\end{enumerate}
\subsubsection{\**hₐel-}
\paragraph{Definition}
\begin{enumerate}
\item to burn
\end{enumerate}
\subsubsection{\**hₐeydʰ-}
\paragraph{Definition}
\begin{enumerate}
\item to burn
\item fire
\end{enumerate}
\subsubsection{\**kehₐw-}
\paragraph{Definition}
\begin{enumerate}
\item to burn
\end{enumerate}
\subsubsection{\**kelh₁-}
\paragraph{Definition}
\begin{enumerate}
\item to incline
\end{enumerate}
\subsubsection{\**ker-}
\paragraph{Definition}
\begin{enumerate}
\item to burn
\end{enumerate}
\subsubsection{\**kʷap-}
\paragraph{Definition}
\begin{enumerate}
\item to smoke
\item to seethe
\end{enumerate}
\subsubsection{\**leh₁w-}
\paragraph{Definition}
\begin{enumerate}
\item stone
\end{enumerate}
\subsubsection{\**lep-}
\paragraph{Definition}
\begin{enumerate}
\item stone
\end{enumerate}
\subsubsection{\**lew-}
\paragraph{Definition}
\begin{enumerate}
\item dirt
\end{enumerate}
\subsubsection{\**may-}
\paragraph{Definition}
\begin{enumerate}
\item soil
\item defile
\end{enumerate}
\subsubsection{\**meh₁-}
\paragraph{Definition}
\begin{enumerate}
\item to measure
\end{enumerate}
\subsubsection{\**meldʰ-}
\paragraph{Definition}
\begin{enumerate}
\item lightning
\end{enumerate}
\subsubsection{\**mregʰ-}
\paragraph{Definition}
\begin{enumerate}
\item to rain softly
\item to drizzle
\end{enumerate}
\subsubsection{\**pen-}
\paragraph{Definition}
\begin{enumerate}
\item water
\end{enumerate}
\subsubsection{\**penk-}
\paragraph{Definition}
\begin{enumerate}
\item wet?
\item mud?
\end{enumerate}
\subsubsection{\**perk-}
\paragraph{Definition}
\begin{enumerate}
\item glowing ash
\item coal
\end{enumerate}
\subsubsection{\**prews-}
\paragraph{Definition}
\begin{enumerate}
\item frost
\end{enumerate}
\subsubsection{\**prews-}
\paragraph{Definition}
\begin{enumerate}
\item to burn
\end{enumerate}
\subsubsection{\**pē(n)s-}
\paragraph{Definition}
\begin{enumerate}
\item dust
\end{enumerate}
\subsubsection{\**snewdʰ-}
\paragraph{Definition}
\begin{enumerate}
\item cloud
\end{enumerate}
\subsubsection{\**sneygʷʰ-}
\paragraph{Definition}
\begin{enumerate}
\item to snow
\end{enumerate}
\subsubsection{\**solhₓ-}
\paragraph{Definition}
\begin{enumerate}
\item dirt
\item dirty
\end{enumerate}
\subsubsection{\**suhₓ-}
\paragraph{Definition}
\begin{enumerate}
\item rain
\end{enumerate}
\subsubsection{\**swel-}
\paragraph{Definition}
\begin{enumerate}
\item to burn
\end{enumerate}
\subsubsection{\**swelp-}
\paragraph{Definition}
\begin{enumerate}
\item to burn
\item to smoulder
\end{enumerate}
\subsubsection{\**tehₐ-}
\paragraph{Definition}
\begin{enumerate}
\item to melt
\end{enumerate}
\subsubsection{\**tenh̥ₐg-}
\paragraph{Definition}
\begin{enumerate}
\item shallow water?
\end{enumerate}
\subsubsection{\**tihₓn-}
\paragraph{Definition}
\begin{enumerate}
\item to be dirty
\end{enumerate}
\subsubsection{\**wehₓp-}
\paragraph{Definition}
\begin{enumerate}
\item body of water
\end{enumerate}
\subsubsection{\**yeg-}
\paragraph{Definition}
\begin{enumerate}
\item ice
\item icicle
\item solid expanse of ice
\end{enumerate}
\subsubsection{\**yuhₓ-}
\paragraph{Definition}
\begin{enumerate}
\item water
\end{enumerate}


\chapter{Words}

\backmatter
\chapter{Index of Roots}

\chapter{Bibliography}

\begin{enumerate}
\item Kroonen, Guus. \textit{Etymological Dictionary of Proto-Germanic}. Leiden: Koninklijke Brill NV, 2013.
\item Mallory, J. P., Adams, D. Q.. \textit{The Oxford Introduction to Proto-Indo-European and the Proto-Indo-European World}. New York: Oxford University Press Inc., 2006.
\end{enumerate}

\end{document}