\documentclass[paper=6in:9in]{scrbook}
%behind the scenes
\usepackage{fontspec}
\usepackage{color}
\usepackage{calc}

%looks
\renewcommand*\sectfont{\normalcolor\bfseries}
\usepackage[hidelinks]{hyperref}

%tables
\usepackage{tabu}
\usepackage{multirow}
\usepackage{multicol}
\usepackage{float}
\restylefloat{table}

%font stuff
\setmainfont[Ligatures=TeX]{Charis SIL}
\newfontfamily\lib{Linux Libertine}

%macros
\usepackage{xebolpie}
\newcommand{\pie}[1]{\PIE{#1}{circumflex, hx}}
\newcommand{\dicttitle}[1]{\PIE{#1}{circumflex, hx, returnvar}\section*{*\apple}}
\newcommand{\formation}[5]{\noindent\-\textsc{#1} #2 \textit{#3} #4 #5}
\newcommand{\subformation}[5]{\\\noindent\-\hspace{2em}\begin{minipage}{\linewidth}\textsc{#1} #2 \textit{#3} #4 #5\end{minipage}}

\title{\PIE{Leg'tik''om}{noasterisk, circumflex}}
\author{A Proto-Indo-European Dictionary}
\date{Author}

\begin{document}
\frontmatter
\maketitle

\newpage

\tableofcontents
\newpage

\chapter{Preface}

\chapter{Abbreviations and Symbols}

\mainmatter
\chapter{Introduction}

\part{Grammar}

\part{Dictionary}

\chapter{Roots}


\dicttitle{solxx} 
\begin{enumerate}
\item{dirt}
\item{dirty}
\end{enumerate}

\noindent\textbf{\PIE{X-wos}{circumflex, hx, noasterisk}}\\
\noindent\-\textsc{PGmc} salwaz \emph{yellowish; brown} \{3\}\\
\noindent\-\hspace{2em}\textsc{Ang} salo\textasciitilde salu \emph{dark; dusky} \{2\} \{3\}\\
\noindent\-\hspace{4em}\textsc{Eng} sallow \{2\} \{3\}\\
\noindent\-\hspace{2em}\textsc{Dum} salu(we) \emph{pale; yelllow; dirty} \{3\}\\
\noindent\-\hspace{2em}\textsc{Goh} salo \emph{dark; black; dirty} \{3\}\\
\noindent\-\hspace{2em}\textsc{Non} sǫlr  \emph{yellow; pale} \{3\}\\
\noindent\textbf{\PIE{?}{circumflex, hx, noasterisk}}\\
\noindent\-\textsc{Ang} sol \emph{dark; dirty} \{2\}\\
\noindent\-\textsc{Lat} salebra \emph{dirt} \{2\}\\
\noindent\-\textsc{Txb} sal \emph{dirty} \{2\}\\
\noindent\-\textsc{Hit} ?salpa- \emph{dog-dung} \{2\}\\


\formation{PGmc}{salwaz}{yellowish; brown}{\{3\}}{
	\subformation{Ang}{salo\textasciitilde salu}{dark; dusky}{\{2\} \{3\}}{
		\subformation{Eng}{sallow}{}{\{2\} \{3\}}{}
	} 
}
\newpage
\chapter{Words}

\backmatter
\chapter{Index of Roots}

\chapter{Bibliography}

\begin{enumerate}
\item Kroonen, Guus. \textit{Etymological Dictionary of Proto-Germanic}. Leiden: Koninklijke Brill NV, 2013.
\item Mallory, J. P., Adams, D. Q.. \textit{The Oxford Introduction to Proto-Indo-European and the Proto-Indo-European World}. New York: Oxford University Press Inc., 2006.
\end{enumerate}

\end{document}